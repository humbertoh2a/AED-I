%%%%%%%%%%%%%%%%%%%%%%%%%%%%%%%%%%%%%%%%%%%%%%%%%%%%%%%%%%%%%%%%%%%%%%%%%%%%%%%%
% ------------------------- PREÂMBULO DO DOCUMENTO --------------------------- %
%%%%%%%%%%%%%%%%%%%%%%%%%%%%%%%%%%%%%%%%%%%%%%%%%%%%%%%%%%%%%%%%%%%%%%%%%%%%%%%%

% Define a classe do documento. 'article' é ideal para relatórios curtos.
% 'a4paper' define o tamanho do papel e '12pt' o tamanho da fonte.
\documentclass[a4paper, 12pt]{article}

% --- Pacotes Essenciais ---

% Codificação de caracteres para aceitar acentos (ç, ã, é, etc.)
\usepackage[utf8]{inputenc}
\usepackage[T1]{fontenc}

% Suporte completo para a língua portuguesa (hifenização, nomes como "Figura", etc.)
\usepackage[brazil]{babel}

% Pacotes para matemática avançada (para escrever as fórmulas de complexidade)
\usepackage{amsmath, amssymb}

% Pacote para incluir imagens (seus gráficos)
\usepackage{graphicx}

% Pacote para configurar as margens da página
\usepackage[a4paper, margin=2.5cm]{geometry}

% Pacote para criar links clicáveis no PDF (no sumário e em referências)
\usepackage{hyperref}
\hypersetup{
    colorlinks=true,
    linkcolor=black,
    urlcolor=blue,
}

% Pacote para ter mais controle sobre a posição de figuras e tabelas
\usepackage{float}


%%%%%%%%%%%%%%%%%%%%%%%%%%%%%%%%%%%%%%%%%%%%%%%%%%%%%%%%%%%%%%%%%%%%%%%%%%%%%%%%
% -------------------------- INFORMAÇÕES DO TÍTULO --------------------------- %
%%%%%%%%%%%%%%%%%%%%%%%%%%%%%%%%%%%%%%%%%%%%%%%%%%%%%%%%%%%%%%%%%%%%%%%%%%%%%%%%

\title{Relatório de Análise de Algoritmos:\\ Inversão, Busca Sequencial e Busca Binária}
\author{
    Eduardo Malafronte Alves de Souza (nusp: 16862798) \\
    Humberto Henrique de Amorim (nusp: 16814612) \\
    Lucas Vinicius da Costa (nusp: 16885265)
}
\date{\today} % \today insere a data atual automaticamente


%%%%%%%%%%%%%%%%%%%%%%%%%%%%%%%%%%%%%%%%%%%%%%%%%%%%%%%%%%%%%%%%%%%%%%%%%%%%%%%%
% ---------------------------- INÍCIO DO DOCUMENTO --------------------------- %
%%%%%%%%%%%%%%%%%%%%%%%%%%%%%%%%%%%%%%%%%%%%%%%%%%%%%%%%%%%%%%%%%%%%%%%%%%%%%%%%
\begin{document}

% Gera a página de título com as informações definidas acima
\maketitle

% Gera o sumário automaticamente a partir das seções
\tableofcontents
\newpage


% --------------------------------- INTRODUÇÃO ------------------------------- %
\section{Introdução}

O objetivo geral deste trabalho é implementar, analisar e comparar a eficiência de diferentes algoritmos: a inversão vetorial, a busca sequencial e a busca binária, tanto em sua forma iterativa quanto recursiva. A análise foi conduzida de forma experimental, através da medição do tempo de execução, e teórica, pela contagem de operações de comparação e atribuição no pior caso. Para os testes, foram utilizadas entradas de 10, 100, 1.000 e 5.000 elementos, e o tempo médio foi calculado a partir de, no mínimo, 100 execuções para garantir a precisão dos resultados.

\newpage

% ------------------------------ ANÁLISE DOS ALGORITMOS ---------------------- %
\section{Análise dos Algoritmos}

Nesta seção, apresentamos os resultados obtidos para cada algoritmo, incluindo o gráfico de tempo de execução e a análise teórica de complexidade no pior caso.

% --- Algoritmo 1: Inversão de Vetor ---
\subsection{Inversão de Vetor}

O gráfico de análise de desempenho do algoritmo de inversão vetorial explicita o caráter linear do crescimento do tempo de execução conforme o aumento do tamanho da entrada, comportamento esse esperado para um algoritmo que precisa percorrer o vetor.

\begin{figure}[H] % [H] tenta forçar a imagem a ficar exatamente aqui
    \centering
    \includegraphics[width=0.8\textwidth]{grafico inversao tempo x entrada individual.png} % <-- SUBSTITUA PELO NOME DO SEU ARQUIVO
    \caption{Tempo de execução médio para o algoritmo de Inversão de Vetor.}
    \label{fig:inversao}
\end{figure}

\textbf{Análise de Complexidade (Pior Caso):}
O algoritmo de inversão percorre o vetor até a sua metade, realizando uma operação de troca a cada iteração. Uma troca envolve 3 atribuições. O laço "for" executa $N/2$ vezes. A contagem de operações é aproximadamente:
$$ T(N) = 5\lfloor\frac{N}{2}\rfloor + 2 $$
Isso resulta em uma complexidade de tempo de $O(N)$.

\textbf{Resultados Experimentais - Operações Médias:}
\begin{table}[H]
\centering
\begin{tabular}{|c|c|}
\hline
\textbf{Tamanho da Entrada (N)} & \textbf{Operações Médias} \\
\hline
10 & 27,00 \\
100 & 252,00 \\
1.000 & 2.502,00 \\
5.000 & 12.502,00 \\
\hline
\end{tabular}
\caption{Número de operações médias para o algoritmo de Inversão de Vetor}
\label{tab:inversao}
\end{table}

A complexidade teórica de $O(N)$ justifica o comportamento linear observado na análise gráfica, confirmando que o tempo de execução do algoritmo escala proporcionalmente com o tamanho da entrada.

\newpage

% --- Algoritmo 2: Busca Sequencial ---
\subsection{Busca Sequencial}

O gráfico de desempenho para o algoritmo de busca sequencial também demonstra um crescimento linear do tempo de execução. Assim como no algoritmo de inversão, este comportamento é esperado, pois, no pior caso, a busca sequencial precisa percorrer todos os N elementos do vetor.

\begin{figure}[H]
    \centering
    \includegraphics[width=0.8\textwidth]{grafico busca sequencial tempo x entrada individual.png} % <-- SUBSTITUA PELO NOME DO SEU ARQUIVO
    \caption{Tempo de execução médio (pior caso) para a Busca Sequencial.}
    \label{fig:sequencial}
\end{figure}

\textbf{Análise de Complexidade (Pior Caso):}
O pior caso ocorre quando o elemento procurado não está no vetor ou é o último. Nesse cenário, o algoritmo realiza uma comparação para cada um dos $N$ elementos do vetor.
$$ T(N) = 3N + 2 $$
Onde temos $N$ comparações no laço, $N$ incrementos do contador, $N$ comparações com o elemento procurado, uma comparação inicial e uma operação de retorno. A complexidade de tempo é, portanto, $O(N)$.

\textbf{Resultados Experimentais - Operações Médias:}
\begin{table}[H]
\centering
\begin{tabular}{|c|c|}
\hline
\textbf{Tamanho da Entrada (N)} & \textbf{Operações Médias} \\
\hline
10 & 32,00 \\
100 & 302,00 \\
1.000 & 3.002,00 \\
5.000 & 15.002,00 \\
\hline
\end{tabular}
\caption{Número de operações médias para o algoritmo de Busca Sequencial}
\label{tab:sequencial}
\end{table}

A complexidade teórica de $O(N)$ justifica o comportamento linear observado na análise
gráfica, confirmando que o tempo de execução do algoritmo escala proporcionalmente com
o tamanho da entrada.

\newpage

% --- Algoritmo 3 e 4: Busca Binária ---
\subsection{Buscas Binárias (Iterativa e Recursiva)}
Os algoritmos de busca binária apresentaram desempenho superior aos algoritmos lineares, confirmando sua característica logarítmica fundamental.

\begin{figure}[H]
    \centering
    % Se tiver dois gráficos, pode colocá-los lado a lado ou um abaixo do outro
    \includegraphics[width=0.8\textwidth]{grafico busca binaria iterativa tempo x entrada individual.png} % <-- SUBSTITUA
    \caption{Tempo de execução médio (pior caso) para a Busca Binária Iterativa.}
    \label{fig:binaria_it}
\end{figure}

\begin{figure}[H]
    \centering
    \includegraphics[width=0.8\textwidth]{grafico busca binaria recursiva individual.png} % <-- SUBSTITUA
    \caption{Tempo de execução médio (pior caso) para a Busca Binária Recursiva.}
    \label{fig:binaria_rec}
\end{figure}

\newpage

\textbf{Análise de Complexidade (Pior Caso):}

\textbf{Busca Binária Iterativa:} A cada iteração, o algoritmo realiza comparações para determinar em qual metade continuar a busca, além de operações de atualização dos índices:
$$ T(N) = 5\lfloor\log_2(N)\rfloor + 3 $$

\textbf{Busca Binária Recursiva:} A implementação recursiva adiciona o overhead das chamadas de função:
$$ T(N) = 4\lfloor\log_2(N)\rfloor + 1 $$

\textbf{Resultados Experimentais - Operações Médias:}

\textbf{Busca Binária Iterativa:}
\begin{table}[H]
\centering
\begin{tabular}{|c|c|}
\hline
\textbf{Tamanho da Entrada (N)} & \textbf{Operações Médias} \\
\hline
10 & 23,00 \\
100 & 38,00 \\
1.000 & 53,00 \\
5.000 & 68,00 \\
\hline
\end{tabular}
\caption{Número de operações médias para a Busca Binária Iterativa}
\label{tab:binaria_it}
\end{table}

\textbf{Busca Binária Recursiva:}
\begin{table}[H]
\centering
\begin{tabular}{|c|c|}
\hline
\textbf{Tamanho da Entrada (N)} & \textbf{Operações Médias} \\
\hline
10 & 21,00 \\
100 & 36,00 \\
1.000 & 51,00 \\
5.000 & 66,00 \\
\hline
\end{tabular}
\caption{Número de operações médias para a Busca Binária Recursiva}
\label{tab:binaria_rec}
\end{table}

Ambas as versões mantêm complexidade de tempo $O(\log N)$. Os dados experimentais confirmam o comportamento logarítmico esperado, com a versão iterativa apresentando desempenho consistentemente similar à recursiva em termos de número de operações.

\newpage

% --------------------------------- CONCLUSÃO -------------------------------- %
\section{Conclusão}

A análise experimental dos quatro algoritmos revelou duas classes distintas de complexidade e desempenho, $O(\log N)$ e $O(N)$. Para visualizar essa diferença de forma direta, o gráfico de desempenho comparativo entre todos os algortimos apresenta as curvas de tempo de execução deles em um único eixo.

\begin{figure}[H] % [H] tenta forçar a imagem a ficar exatamente aqui
    \centering
    \includegraphics[width=0.9\textwidth]{grafico comparacao geral desempenho tempo x entrada.png} % <-- SUBSTITUA PELO NOME DO SEU GRÁFICO COMPARATIVO
    \caption{Comparativo de desempenho entre os algoritmos analisados (Pior caso).}
    \label{fig:comparativo}
\end{figure}

O gráfico indica claramente que as curvas dos algoritmos de Inversão de Vetor e Busca Sequencial crescem de forma linear e acentuada, confirmando sua complexidade $O(N)$. Em contraste, as curvas das Buscas Binárias (Iterativa e Recursiva) permanecem praticamente planas na base do gráfico, mal registrando aumento de tempo mesmo com 5.000 elementos. Essa representação visual demonstra a imensa escalabilidade e eficiência dos algoritmos de complexidade logarítmica ($O(\log N)$) em comparação com os lineares ($O(N)$).

Para entradas pequenas, a diferença de tempo pode ser irrelevante, mas para sistemas que manipulam grandes quantidades de dados, a escolha de um algoritmo $O(\log N)$ em vez de um $O(N)$ é de suma importância para a performance. Portanto, é possível concluir que os resultados experimentais obtidos foram consistentes com a teoria de análise de algoritmos, demonstrando a importância do estudo da complexidade para o desenvolvimento de soluções computacionais eficientes.

\newpage

% ------------------------- CONTRIBUIÇÃO DOS MEMBROS ------------------------- %
\section{Contribuição dos Membros da Equipe}

A seguir, a descrição das principais contribuições de cada membro da equipe:

\begin{itemize}
    \item \textbf{Eduardo Malafronte Alves de Souza:}
        \begin{itemize}
            \item Criação de todos os gráficos para a visualização e análise do tempo de execução dos algoritmos.
            \item Redação, estruturação e formatação do relatório, consolidando os resultados e as análises da equipe.
        \end{itemize}
        
    \item \textbf{Humberto Henrique de Amorim:}
        \begin{itemize}
            \item Coleta e tabulação dos dados de tempo de execução.
            \item Análise teórica da contagem de operações dos algoritmos.
        \end{itemize}
        
    \item \textbf{Lucas Vinicius da Costa:}
        \begin{itemize}
            \item Responsável pelo desenvolvimento da solução em linguagem C, implementando os quatro algoritmos solicitados.
        \end{itemize}
\end{itemize}

\end{document}
%%%%%%%%%%%%%%%%%%%%%%%%%%%%%%%%%%%%%%%%%%%%%%%%%%%%%%%%%%%%%%%%%%%%%%%%%%%%%%%%
% ------------------------------ FIM DO DOCUMENTO ---------------------------- %
%%%%%%%%%%%%%%%%%%%%%%%%%%%%%%%%%%%%%%%%%%%%%%%%%%%%%%%%%%%%%%%%%%%%%%%%%%%%%%%%